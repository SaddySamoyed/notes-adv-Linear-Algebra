\documentclass[lang=cn,11pt]{elegantbook}
\usepackage[utf8]{inputenc}
\usepackage[UTF8]{ctex}
\usepackage{amsmath}%
\usepackage{amssymb}%

\title{Some Linear Algebra}
\subtitle{taken from LADR and GTM135}
\author{Qiulin Fan}
\date{2024}

\extrainfo{Mainly intended to serve the convenience of Analysis.}

\logo{assets/M.jpg}
\cover{assets/M.jpg}

% modify the color in the middle of titlepage
\definecolor{customcolor}{RGB}{32,178,170}
\colorlet{coverlinecolor}{customcolor}

\begin{document}
\maketitle
\frontmatter
\tableofcontents

\mainmatter
\chapter{Review on Basic Concepts}
\section{Subspace and direct sum}
\begin{definition}{subsapce} \label{subspace}
vector space 的 subset $U \subset V$ 为一个 subspace,if 它满足条件:
\begin{enumerate}
    \item 包含 0
    \item 对 addition 和 scalar multiplication 闭合
\end{enumerate}
\end{definition}
两个 subset 的和就是各取一个元素相加的所有情况.\\
很显然我们知道:
\begin{proposition}
两个 subspace $U_1, U_2$ 的 sum $U_1 + U_2$ 也是一个 subspace, 并且 
$$
dim(U_1 + U_2) \leq dim(U_1) + dim(U_2)
$$
且 $U_1 + U_2$ 是同时包含 $U_1$ 和 $U_2$ 的 $V $ 的最小 subspace.
\end{proposition}

显然可以随便和。同一个 $U$ 自己和自己的和就是自己。所以 subspace sum 这个概念比较大,没什么用。我们需要用 direct sum 来作为一个小一点但是更有用的概念,表达出一种垂直的 subspace 的直观.

\begin{definition}{direct sum} \label{direct sum}
如果 $U_1 + U_2 + \cdots + \U_m$ 中的任意元素 $v$,都存在唯一的 $v_k \in U_k$ for each $k$ 使得 $v = \sum_{k}v_k$,就称 $U_1 + \cdots + U_m = \bigoplus_{i = 1}^m U_i$ 为一个 direct sum.
\end{definition}

我们显然发现:
\begin{proposition}
    $$
    \text{dim}(\bigoplus_{i = 1}^m U_i) = \sum_{i = 1}^m \text{dim}(U_i) 
    $$
\end{proposition}

我们发现,其实可以 direct sum 的 subspaces 是 “垂直的”,意思是:
\begin{theorem}
    $U_1 + U_2 + \cdots + U_m$ 是一个 direct sum (这几个空间"垂直") iff 任取 $u_1, u_2, \cdots, u_m$ 分别来自 $U_1,  U_2 , \cdots, U_m$,它们都 lin. ind.
\end{theorem}
并且:
\begin{theorem}
    $U_1 + U_2$ 为一个 direct sum iff $U_1 \bigcap U_2 = \{0\}$.
\end{theorem}
\begin{remark}
实际上两个 subspace 的交集里只要有一个非 0 点,那么这个点 span 的整个 dim 为 1 的线都在交集里.
\end{remark}

\begin{note}
    $$
    \mathbb{F}^n = \bigoplus_{i =1}^n \text{span}(e_i)
    $$
\end{note}


\chapter{Linear functional and Duality}
\begin{definition}{Linear functional}
\end{definition}


\chapter{Eigenvalues and Operators}


\chapter{Operators on complex VS}


\chapter{Multilinear Algebra}

\end{document}